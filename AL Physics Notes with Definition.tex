\documentclass[]{article}
\usepackage{lmodern}
\usepackage{amssymb,amsmath}
\usepackage{ifxetex,ifluatex}
\usepackage{fixltx2e} % provides \textsubscript
\ifnum 0\ifxetex 1\fi\ifluatex 1\fi=0 % if pdftex
  \usepackage[T1]{fontenc}
  \usepackage[utf8]{inputenc}
\else % if luatex or xelatex
  \ifxetex
    \usepackage{mathspec}
  \else
    \usepackage{fontspec}
  \fi
  \defaultfontfeatures{Ligatures=TeX,Scale=MatchLowercase}
\fi
% use upquote if available, for straight quotes in verbatim environments
\IfFileExists{upquote.sty}{\usepackage{upquote}}{}
% use microtype if available
\IfFileExists{microtype.sty}{%
\usepackage[]{microtype}
\UseMicrotypeSet[protrusion]{basicmath} % disable protrusion for tt fonts
}{}
\PassOptionsToPackage{hyphens}{url} % url is loaded by hyperref
\usepackage[unicode=true]{hyperref}
\hypersetup{
            pdfborder={0 0 0},
            breaklinks=true}
\urlstyle{same}  % don't use monospace font for urls
\IfFileExists{parskip.sty}{%
\usepackage{parskip}
}{% else
\setlength{\parindent}{0pt}
\setlength{\parskip}{6pt plus 2pt minus 1pt}
}
\setlength{\emergencystretch}{3em}  % prevent overfull lines
\providecommand{\tightlist}{%
  \setlength{\itemsep}{0pt}\setlength{\parskip}{0pt}}
\setcounter{secnumdepth}{0}
% Redefines (sub)paragraphs to behave more like sections
\ifx\paragraph\undefined\else
\let\oldparagraph\paragraph
\renewcommand{\paragraph}[1]{\oldparagraph{#1}\mbox{}}
\fi
\ifx\subparagraph\undefined\else
\let\oldsubparagraph\subparagraph
\renewcommand{\subparagraph}[1]{\oldsubparagraph{#1}\mbox{}}
\fi

% set default figure placement to htbp
\makeatletter
\def\fps@figure{htbp}
\makeatother


\date{}

\begin{document}

\section{AL Physics Notes}\label{header-n0}

\subsection{Circular Motions}\label{header-n2}

\subsubsection{Definition}\label{header-n3}

\begin{itemize}
\item
  Uniform Circular Motion:

  \begin{itemize}
  \item
    motion going around the circle at constant speed
  \end{itemize}
\item
  Radian:

  \begin{itemize}
  \item
    One radian is the angle \textbf{subtended} at the centre of a circle
    by an arc of length equal the \textbf{radius} of the circle
  \item
    \(1 \space rad = \cfrac{180^{\circ}}{\pi}\)
  \item
    \(1^{\circ} = \cfrac{\pi}{180^{\circ}}\)
  \end{itemize}
\item
  Angular displacement:

  \begin{itemize}
  \item
    angle \(\theta\) of rotation if the angle is \textbf{in radian}
  \end{itemize}
\item
  Angular velocity:

  \begin{itemize}
  \item
    angular speed is the rate of change of angle \textbf{in radian}
  \item
    \(\omega = \cfrac{\Delta \theta}{\Delta t}= \cfrac{2 \pi}{T} \)
  \end{itemize}
\item
  Period

  \begin{itemize}
  \item
    \(T\) time for make one \textbf{complete} revolution
  \item
    \(T=\cfrac{1}{f}\)
  \end{itemize}
\item
  Frequency

  \begin{itemize}
  \item
    \(f\) number of revolution per second
  \item
    \(f=\cfrac{1}{T}\)
  \end{itemize}
\end{itemize}

\subsubsection{Key Points}\label{header-n45}

\begin{itemize}
\item
  Linear velocity:

  \begin{itemize}
  \item
    \(v =\cfrac{s}{t} = \cfrac{2\pi r}{T} =2\pi rf = r\omega \)
  \item
    if not specify, speed for UCM means linear speed
  \item
    numerical value for velocity is the same as speed, the direction is
    always the tangent with the direction of moving
  \end{itemize}
\item
  Acceleration

  \begin{itemize}
  \item
    \(a = \cfrac{\overrightarrow{\rm \Delta v}}{\Delta t} =  \cfrac{v^2}{r} = r\omega^2\)
  \item
    notice that the velocity here used to derive the acceleration must
    have the vector notation
  \item
    derivation obtained by consider isosceles triangle and take ratio
    for similar triangle. Cord approximate to arc and obtain the answer
    under the consideration of small amount of time
  \end{itemize}
\item
  Centripetal force

  \begin{itemize}
  \item
    \(F = \cfrac{mv^2}{r} = mr\omega^2\)
  \item
    tension changes during rotation vertically
  \item
    at the top, \(T+mg = \cfrac{mv^2}{r}\)
  \item
    at the bottom, \(T-mg = \cfrac{mv^2}{r}\)
  \end{itemize}
\item
  Energy \& Work

  \begin{itemize}
  \item
    Work = 0
  \item
    Energy not used
  \item
    \(W = Fd \cos{\theta}\) since the direction of movement is always
    perpendicular so no work is done, hence no energy is used
  \end{itemize}
\item
  Origin of centripetal force

  \begin{itemize}
  \item
    resolving the vector results in two direction of force, and that
    will be one which provide centripetal force
  \end{itemize}
\end{itemize}

\subsection{Oscillations}\label{header-n90}

\subsubsection{Definition}\label{header-n91}

\begin{itemize}
\item
  Angular frequency:

  \begin{itemize}
  \item
    number of oscillation made in 1s
  \end{itemize}
\item
  Phase difference:

  \begin{itemize}
  \item
    phase difference for same curve, two point:
    \(\phi= \cfrac{2\pi}{T} * \Delta t\)
  \item
    phase difference for different curve: draw vertical line and compare
  \end{itemize}
\item
  SHM

  \begin{itemize}
  \item
    type f motion when at \textbf{any moment of time}, a is
    \textbf{proportional} to x in \textbf{opposite direction} and always
    \textbf{towards e.p}.
  \item
    \(a = -\omega^2 x\)
  \end{itemize}
\item
  Oscillations

  \begin{itemize}
  \item
    Free Oscillation - harmonic

    \begin{itemize}
    \item
      the only external force acting on it is the restoring force
    \item
      it vibrates at its natural frequency(\(f_0\))
    \end{itemize}
  \item
    Forced Oscillation - non harmonic

    \begin{itemize}
    \item
      there is external driving force with driving frequency acting
      periodically on the oscillation
    \end{itemize}
  \end{itemize}

  \begin{itemize}
  \item
    Damped Oscillation

    \begin{itemize}
    \item
      External resistive and frictional force cause the oscillator's
      energy to dissipate into heat
    \item
      amplitude will decreased
    \item
      A-f diagram will not be sharp
    \end{itemize}
  \end{itemize}
\item
  Resounance

  \begin{itemize}
  \item
    Natural frequency is equal to the frequency of the driver
  \item
    Amplitude is maximum (dramaticaly increased amplitude)
  \item
    Absorbs the greatest possible energy from the driver
  \end{itemize}
\end{itemize}

\subsubsection{key Points}\label{header-n110}

Formula for SHM

\begin{itemize}
\item
  \(x = X_0 \sin{\omega t}\)
\item
  \(v = \omega X_0 \cos{\omega t}\)
\item
  \(a = -\omega^2 X_0 sin{\omega t}\)
\end{itemize}

Formula in SHM

\begin{itemize}
\item
  \(v_{max} = x_0\omega\)
\item
  \(a_{max} = x_0 \omega^2\)
\item
  \(v = v_0\cos{\omega t}  = \omega \sqrt{x_0^2 - x^2}\)
\end{itemize}

Period of mass-spring system

\begin{itemize}
\item
  \(T = 2\pi\sqrt{\cfrac{m}{k}}\)
\end{itemize}

Period of mathematical simple pendulum

\begin{itemize}
\item
  \(T  = 2\pi \sqrt{\cfrac{l}{g}}\)
\item
  mass of the bulb is larger than mass of string
\item
  size of the bulb is negligible compare with length of string
\item
  angle should be small
\end{itemize}

Energy changes during oscillation

\begin{itemize}
\item
  \(E_k = \cfrac{1}{2}m\omega^2(x_0^2 - x^2)\), maximum when going
  through e.p.
\item
  \(E_p = \cfrac{1}{2}m\omega^2x^2\), maximum when at the two ends of
  oscillation
\item
  \(E_t = \frac{1}{2}m\omega^2x_0^2\), note the total energy don't
  change
\end{itemize}

\subsection{Gravitational Fields}\label{header-n126}

\subsection{Electric Fields}\label{header-n128}

\subsection{Magnetic Fields}\label{header-n130}

\subsection{Electromagnetism}\label{header-n132}

\subsection{Alternative Current}\label{header-n134}

\subsection{Capacitance}\label{header-n136}

\subsection{Electronics}\label{header-n138}

\subsection{Thermodynamics}\label{header-n140}

\subsection{Communication}\label{header-n142}

\subsection{Quantum Physics}\label{header-n144}

\subsection{Nuclear Physics}\label{header-n146}

\subsection{Medical Imaging}\label{header-n148}

\(\Epsilon \Delta\Xi\)

\end{document}
