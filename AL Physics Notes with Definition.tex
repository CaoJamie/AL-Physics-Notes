\documentclass[]{article}
\usepackage{lmodern}
\usepackage{amssymb,amsmath}
\usepackage{ifxetex,ifluatex}
\usepackage{fixltx2e} % provides \textsubscript
\ifnum 0\ifxetex 1\fi\ifluatex 1\fi=0 % if pdftex
  \usepackage[T1]{fontenc}
  \usepackage[utf8]{inputenc}
\else % if luatex or xelatex
  \ifxetex
    \usepackage{mathspec}
  \else
    \usepackage{fontspec}
  \fi
  \defaultfontfeatures{Ligatures=TeX,Scale=MatchLowercase}
\fi
% use upquote if available, for straight quotes in verbatim environments
\IfFileExists{upquote.sty}{\usepackage{upquote}}{}
% use microtype if available
\IfFileExists{microtype.sty}{%
\usepackage[]{microtype}
\UseMicrotypeSet[protrusion]{basicmath} % disable protrusion for tt fonts
}{}
\PassOptionsToPackage{hyphens}{url} % url is loaded by hyperref
\usepackage[unicode=true]{hyperref}
\hypersetup{
            pdfborder={0 0 0},
            breaklinks=true}
\urlstyle{same}  % don't use monospace font for urls
\IfFileExists{parskip.sty}{%
\usepackage{parskip}
}{% else
\setlength{\parindent}{0pt}
\setlength{\parskip}{6pt plus 2pt minus 1pt}
}
\setlength{\emergencystretch}{3em}  % prevent overfull lines
\providecommand{\tightlist}{%
  \setlength{\itemsep}{0pt}\setlength{\parskip}{0pt}}
\setcounter{secnumdepth}{0}
% Redefines (sub)paragraphs to behave more like sections
\ifx\paragraph\undefined\else
\let\oldparagraph\paragraph
\renewcommand{\paragraph}[1]{\oldparagraph{#1}\mbox{}}
\fi
\ifx\subparagraph\undefined\else
\let\oldsubparagraph\subparagraph
\renewcommand{\subparagraph}[1]{\oldsubparagraph{#1}\mbox{}}
\fi

% set default figure placement to htbp
\makeatletter
\def\fps@figure{htbp}
\makeatother


\date{}

\begin{document}

\section{AL Physics Notes}\label{header-n0}

\subsection{Circular Motions}\label{header-n2}

\subsubsection{Definition}\label{header-n3}

\begin{itemize}
\item
  Uniform Circular Motion:

  \begin{itemize}
  \item
    motion going around the circle at constant speed
  \end{itemize}
\item
  Radian:

  \begin{itemize}
  \item
    One radian is the angle \textbf{subtended} at the centre of a circle
    by an arc of length equal the \textbf{radius} of the circle
  \item
    \(1 \space rad = \cfrac{180^{\circ}}{\pi}\)
  \item
    \(1^{\circ} = \cfrac{\pi}{180^{\circ}}\)
  \end{itemize}
\item
  Angular displacement:

  \begin{itemize}
  \item
    angle \(\theta\) of rotation if the angle is \textbf{in radian}
  \end{itemize}
\item
  Angular velocity:

  \begin{itemize}
  \item
    angular speed is the rate of change of angle \textbf{in radian}
  \item
    \(\omega = \cfrac{\Delta \theta}{\Delta t}= \cfrac{2 \pi}{T} \)
  \end{itemize}
\item
  Period

  \begin{itemize}
  \item
    \(T\) time for make one \textbf{complete} revolution
  \item
    \(T=\cfrac{1}{f}\)
  \end{itemize}
\item
  Frequency

  \begin{itemize}
  \item
    \(f\) number of revolution per second
  \item
    \(f=\cfrac{1}{T}\)
  \end{itemize}
\end{itemize}

\subsubsection{Key points}\label{header-n45}

\begin{itemize}
\item
  Linear velocity:

  \begin{itemize}
  \item
    \(v =\cfrac{s}{t} = \cfrac{2\pi r}{T} =2\pi rf = r\omega \)
  \item
    if not specify, speed for UCM means linear speed
  \item
    numerical value for velocity is the same as speed, the direction is
    always the tangent with the direction of moving
  \end{itemize}
\item
  Acceleration

  \begin{itemize}
  \item
    \(a = \cfrac{\overrightarrow{\rm \Delta v}}{\Delta t} =  \cfrac{v^2}{r} = r\omega^2\)
  \item
    notice that the velocity here used to derive the acceleration must
    have the vector notation
  \item
    derivation obtained by consider isosceles triangle and take ratio
    for similar triangle. Cord approximate to arc and obtain the answer
    under the consideration of small amount of time
  \end{itemize}
\item
  Centripetal force

  \begin{itemize}
  \item
    \(F = \cfrac{mv^2}{r} = mr\omega^2\)
  \item
    tension changes during rotation vertically
  \item
    at the top, \(T+mg = \cfrac{mv^2}{r}\)
  \item
    at the bottom, \(T-mg = \cfrac{mv^2}{r}\)
  \end{itemize}
\item
  Energy \& Work

  \begin{itemize}
  \item
    Work = 0
  \item
    Energy not used
  \item
    \(W = Fd \cos{\theta}\) since the direction of movement is always
    perpendicular so no work is done, hence no energy is used
  \end{itemize}
\item
  Origin of centripetal force

  \begin{itemize}
  \item
    resolving the vector results in two direction of force, and that
    will be one which provide centripetal force
  \end{itemize}
\end{itemize}

\subsection{Oscillations}\label{header-n90}

\subsubsection{Definition}\label{header-n91}

\subsection{Gravitational Fields}\label{header-n93}

\subsection{Electric Fields}\label{header-n95}

\subsection{Magnetic Fields}\label{header-n97}

\subsection{Electromagnetism}\label{header-n99}

\subsection{Alternative Current}\label{header-n101}

\subsection{Capacitance}\label{header-n103}

\subsection{Electronics}\label{header-n105}

\subsection{Thermodynamics}\label{header-n107}

\subsection{Communication}\label{header-n109}

\subsection{Quantum Physics}\label{header-n111}

\subsection{Nuclear Physics}\label{header-n113}

\subsection{Medical Imaging}\label{header-n115}

\(\Epsilon \Delta\Xi\)

\end{document}
